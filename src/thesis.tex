%%% Hlavní soubor. Zde se definují základní parametry a odkazuje se na ostatní části. %%%

%% Verze pro jednostranný tisk:
% Okraje: levý 40mm, pravý 25mm, horní a dolní 25mm
% (ale pozor, LaTeX si sám přidává 1in)
\documentclass[12pt,a4paper]{report}
\setlength\textwidth{145mm}
\setlength\textheight{247mm}
\setlength\oddsidemargin{15mm}
\setlength\evensidemargin{15mm}
\setlength\topmargin{0mm}
\setlength\headsep{0mm}
\setlength\headheight{0mm}
% \openright zařídí, aby následující text začínal na pravé straně knihy
\let\openright=\clearpage

%% Pokud tiskneme oboustranně:
% \documentclass[12pt,a4paper,twoside,openright]{report}
% \setlength\textwidth{145mm}
% \setlength\textheight{247mm}
% \setlength\oddsidemargin{15mm}
% \setlength\evensidemargin{0mm}
% \setlength\topmargin{0mm}
% \setlength\headsep{0mm}
% \setlength\headheight{0mm}
% \let\openright=\cleardoublepage

%% Použité kódování znaků: obvykle latin2, cp1250 nebo utf8:
\usepackage[utf8]{inputenc}

%% Ostatní balíčky
\usepackage{graphicx}
\usepackage{amsthm}
\usepackage{listings}

%% Balíček hyperref, kterým jdou vyrábět klikací odkazy v PDF,
%% ale hlavně ho používáme k uložení metadat do PDF (včetně obsahu).
%% POZOR, nezapomeňte vyplnit jméno práce a autora.
\usepackage[ps2pdf,unicode]{hyperref}   % Musí být za všemi ostatními balíčky
\hypersetup{pdftitle=Querying NoSQL databases in MPS}
\hypersetup{pdfauthor=Radovan Duga}

%%% Drobné úpravy stylu

% Tato makra přesvědčují mírně ošklivým trikem LaTeX, aby hlavičky kapitol
% sázel příčetněji a nevynechával nad nimi spoustu místa. Směle ignorujte.
\makeatletter
\def\@makechapterhead#1{
  {\parindent \z@ \raggedright \normalfont
   \Huge\bfseries \thechapter. #1
   \par\nobreak
   \vskip 20\p@
}}
\def\@makeschapterhead#1{
  {\parindent \z@ \raggedright \normalfont
   \Huge\bfseries #1
   \par\nobreak
   \vskip 20\p@
}}
\makeatother

% Toto makro definuje kapitolu, která není očíslovaná, ale je uvedena v obsahu.
\def\chapwithtoc#1{
\chapter*{#1}
\addcontentsline{toc}{chapter}{#1}
}

\begin{document}

% Trochu volnější nastavení dělení slov, než je default.
\lefthyphenmin=2
\righthyphenmin=2

%%% Titulní strana práce

\pagestyle{empty}
\begin{center}

\large

Charles University in Prague

\medskip

Faculty of Mathematics and Physics

\vfill

{\bf\Large MASTER THESIS}

\vfill

\centerline{\mbox{\includegraphics[width=60mm]{img/logo.eps}}}

\vfill
\vspace{5mm}

{\LARGE Bc. Radovan Duga}

\vspace{15mm}

% Název práce přesně podle zadání
{\LARGE\bfseries Querying NoSQL databases in MPS}

\vfill

% Název katedry nebo ústavu, kde byla práce oficiálně zadána
% (dle Organizační struktury MFF UK)
Department of Distributed and Dependable Systems

\vfill

\begin{tabular}{rl}

Supervisor of the master thesis: & RNDr. Pavel Parízek, Ph.D. \\
\noalign{\vspace{2mm}}
Study programme: & Computer Science \\
\noalign{\vspace{2mm}}
Specialization: & Software Systems \\
\end{tabular}

\vfill

% Zde doplňte rok
Prague 2014

\end{center}


\newpage

%%% Následuje vevázaný list -- kopie podepsaného "Zadání diplomové práce".
%%% Toto zadání NENÍ součástí elektronické verze práce, nescanovat.

%%% Na tomto místě mohou být napsána případná poděkování (vedoucímu práce,
%%% konzultantovi, tomu, kdo zapůjčil software, literaturu apod.)

\openright

\noindent
Dedication.

% TODO: dedication

\newpage

%%% Strana s čestným prohlášením k diplomové práci

\vglue 0pt plus 1fill

\noindent
I declare that I carried out this master thesis independently, and only with the cited
sources, literature and other professional sources.

\medskip\noindent
I understand that my work relates to the rights and obligations under the Act No.
121/2000 Coll., the Copyright Act, as amended, in particular the fact that the Charles
University in Prague has the right to conclude a license agreement on the use of this
work as a school work pursuant to Section 60 paragraph 1 of the Copyright Act.

\vspace{10mm}

\hbox{\hbox to 0.5\hsize{%
In ........ date ............
\hss}\hbox to 0.5\hsize{%
signature of the author
\hss}}

\vspace{20mm}
\newpage

%%% Povinná informační strana diplomové práce

\vbox to 0.5\vsize{
\setlength\parindent{0mm}
\setlength\parskip{5mm}

Název práce:
Dotazování NoSQL databáz v prostředí MPS
% přesně dle zadání

Autor:
Radovan Duga

Katedra:  % Případně Ústav:
Katedra distribuovaných a spolehlivých systémů
% dle Organizační struktury MFF UK

Vedoucí diplomové práce:
RNDr. Pavel Parízek, Ph.D., Katedra distribuovaných a spolehlivých systémů
% dle Organizační struktury MFF UK, případně plný název pracoviště mimo MFF UK

Abstrakt:
% abstrakt v rozsahu 80-200 slov; nejedná se však o opis zadání diplomové práce
% TODO: cz change assignment -> abstract
S příchodem NoSQL databází se objevila i potřeba pro vznik doménově specifických dotazovacích
jazyků. Jednou ze zajímavých domén jsou grafové databáze jako například Neo4j s dotazovací jazykem
Cypher. Doménově specifické jazyky (DSLs) může být navržena a snadno použita pomocí speciálních
vývojových prostředích zvaných Language Workbenche. Velmi populární Language Workbench je MPS,
který implementuje koncept projekčních DSLs.

Tato práce zodpovídá otázku, zda Language Workbenche a projekční DSLs mohou být přínosem v
doméně NoSQL databází, vystihnout výhody projekčních DSLs použitím různých typů přístupu. Dalším
specifickým cílem je navrhnout a implementovat dotazovací DSL jazyk pro vybranou NoSQL databázi
(např. Neo4J nebo Redis) jako případová studie.

Klíčová slova:
NoSQL, MPS, dotaz, Cypher
% 3 až 5 klíčových slov

\vss}\nobreak\vbox to 0.49\vsize{
\setlength\parindent{0mm}
\setlength\parskip{5mm}

Title:
Querying NoSQL databases in MPS
% přesný překlad názvu práce v angličtině

Author:
Radovan Duga

Department:
Department of Distributed and Dependable Systems
% dle Organizační struktury MFF UK v angličtině

Supervisor:
RNDr. Pavel Parízek, Ph.D., Department of Distributed and Dependable Systems
% dle Organizační struktury MFF UK, případně plný název pracoviště
% mimo MFF UK v angličtině

Abstract:
% abstrakt v rozsahu 80-200 slov v angličtině; nejedná se však o překlad
% zadání diplomové práce
% TODO: en change assignment -> abstract
With the advent of NoSQL databases, a need for targeted domain-specific query languages has become
evident. One of the interesting domains are graph databases, such as Neo4j with the query language
Cypher. Domain specific languages (DSLs) can be designed and easily used with the help of special
development environments called Language Workbenches. A very popular Language Workbench is MPS,
which implements the concept of projectional DSLs.

This work will answer the question whether Language Workbenches and projectional DSLs can make a
contribution in the domain of NoSQL databases, and identify the benefits of projectional DSLs over
different approaches. An additional specific goal is to design and implement a practical MPS-based
query DSL for a chosen NoSQL database (e.g., Neo4J or Redis) as a case study.

Keywords:
NoSQL, MPS, query, Cypher
% 3 až 5 klíčových slov v angličtině

\vss}

\newpage

%%% Strana s automaticky generovaným obsahem diplomové práce. U matematických
%%% prací je přípustné, aby seznam tabulek a zkratek, existují-li, byl umístěn
%%% na začátku práce, místo na jejím konci.

\openright
\pagestyle{plain}
\setcounter{page}{1}
\tableofcontents

%%% Jednotlivé kapitoly práce jsou pro přehlednost uloženy v samostatných souborech
\chapter{Introduction}
% \addcontentsline{toc}{chapter}{Introduction}

\section{Motivation}

% TODO: proc je uzitecne implementovat MPSypher (zkusit pouziti DSL a MPS)

\section{Goals}

% TODO: zadani diplomky

The goals of this thesis are these:

\begin{itemize}
  \item item1
  \item item2
\end{itemize}
%%%%%%%%%%%%%%%%%%%%%%%%%%%%%%%%%%%%%
\chapter{Background}

\section{NoSQL databases}

	\subsection{Neo4j graph database}

	\subsection{Neo4j Cypher query language}
	
%popsat jazyk Cypher (syntax, semantika, priklady, atd)

\section{Domain specific languages}

	\subsection{Neo4jCypher DSL}



\section{Language workbenches}

	\subsection{How to define DSL in MPS}



\section{MPS Language workbench}

	\subsection{How to define DSL in MPS}

		\subsubsection{Structure DSL}
		\subsubsection{Editor DSL}
		\subsubsection{Constraints DSL}
		\subsubsection{Typesystem DSL}
		\subsubsection{Intensions and other parts}

	\subsection{MPS Pros and Cons}


%%%%%%%%%%%%%%%%%%%%%%%%%%%%%%%%%%%%%
\chapter{Design of Neo4jCypher}

\section{Problem analysis}

\section{Design decisions}

\subsection{Projection editor}

MPS Projection editor is a base approach of MPS to editing language. It differs from pure textual
editors in these points:

\begin{itemize}
  \item Unparsable editor notation - editor is composed of cells, every cell represents some node of
  the abstract syntax tree (AST)
  \item Possible switching among multiple notations using alternative editors, where every editor
  can show the AST in different way
  \item Combine notations from different authors and be extensible. It is possible to extend MPS
  languages, compose them and so on
\end{itemize}

\subsection{Text-like editor}

\subsection{Graphical extensions}


%%%%%%%%%%%%%%%%%%%%%%%%%%%%%%%%%%%%%
\chapter{Implementation details of Neo4jCypher language}

\section{Patterns}

\section{References}

\section{Integration into BaseLanguage}

MPS has a big advantage that we not only can design the new language and create the full ide
support for this language but also we can integrate this language into some existing language,
extend it and provide the way to execute our code with possibility to debug our code.

We chose these 4 steps, where each next step integrates the Neo4jCypher language more closely into
its BaseLanguage:

\begin{itemize}
  \item Design Neo4jCypher queries in Query Sheet editor - editor with full ide support:
  	\begin{itemize}
  	  \item code completition with correct reference visibility and full cypher function list
  	  \item intensions for query transformation, parts of query grouping, relationship
  	  type conversion, properties definition
  	  \item identifier uniquiness checking
  	  \item items type checking where necessary
  	\end{itemize} 
  \item Generation of MPCypher query into BaseLanguage string type - this approach provides us
  easy creation of query with full ide editor support. The next query manipulation depends on user's
  will. User can store this query or execute it using Neo4j Execution Engine from Neo4j java
  library.
  \item Using CypherExecutor object which adds us easy access to execution of Neo4jCypher query
  without need of implementation any further code from Neo4j java library. This approach tries to
  provide very easy way to query the database without need to know any king of Cypher java
  interface.
\end{itemize}



%%%%%%%%%%%%%%%%%%%%%%%%%%%%%%%%%%%%%
\chapter{Evaluation}

\section{Experience with MPS}

\section{MPS Contribution in DSL languages}

% Pros&Cons

\section{MPS Contribution in NoSQL Domain}

\section{Related work}
		
\section{Case Study}

% What to study in the future, next ideas of the work, improvements








\include{chap3_design}
\include{chap4_implementation}
\include{chap5_evaluation}

% Ukázka použití některých konstrukcí LateXu (odkomentujte, chcete-li)
% \include{example}

\include{epilog}

%%% Seznam použité literatury
%%% Seznam použité literatury je zpracován podle platných standardů. Povinnou citační
%%% normou pro diplomovou práci je ISO 690. Jména časopisů lze uvádět zkráceně, ale jen
%%% v kodifikované podobě. Všechny použité zdroje a prameny musí být řádně citovány.

\def\bibname{Bibliography}
\begin{thebibliography}{99}
\addcontentsline{toc}{chapter}{\bibname}

%\bibitem{lamport94}
%  {\sc Lamport,} Leslie.
%  \emph{\LaTeX: A Document Preparation System}.
%  2. vydání.
%  Massachusetts: Addison Wesley, 1994.
%  ISBN 0-201-52983-1.

\bibitem{lamport94}
  {\sc Fowler,} Martin.
  \emph{Domain-Specific Languages}.
  Massachusetts: Addison Wesley, 2013.
  ISBN 0-321-71294-3.

\bibitem{lamport94}
  {\sc JetBrains}.
  \emph{MPS Documents and Live Demos}.
  \url{http://www.jetbrains.com/mps/documentation/index.html}.

\bibitem{lamport94}
  {\sc JetBrains}.
  \emph{MPS User's Guide}.
  \url{http://confluence.jetbrains.com/display/\\MPSD25/MPS+User's+Guide}.
  
\bibitem{lamport94}
  {\sc Voelter} Marcus.
  \emph{Language and IDE Modularization, Extension and Composition with MPS}.
  \url{http://voelter.de/data/pub/Voelter-GTTSE-MPS.pdf}.
  
\end{thebibliography}


%%% Tabulky v diplomové práci, existují-li.
\chapwithtoc{List of Tables}

%%% Použité zkratky v diplomové práci, existují-li, včetně jejich vysvětlení.
\chapwithtoc{List of Abbreviations}

%%% Přílohy k diplomové práci, existují-li (různé dodatky jako výpisy programů,
%%% diagramy apod.). Každá příloha musí být alespoň jednou odkazována z vlastního
%%% textu práce. Přílohy se číslují.
\chapwithtoc{Attachments}

\openright
\end{document}
