%%% Na tomto místě mohou být napsána případná poděkování (vedoucímu práce,
%%% konzultantovi, tomu, kdo zapůjčil software, literaturu apod.)

\openright

\noindent
Dedication.

% TODO: dedication

\newpage

%%% Strana s čestným prohlášením k diplomové práci

\vglue 0pt plus 1fill

\noindent
I declare that I carried out this master thesis independently, and only with the cited
sources, literature and other professional sources.

\medskip\noindent
I understand that my work relates to the rights and obligations under the Act No.
121/2000 Coll., the Copyright Act, as amended, in particular the fact that the Charles
University in Prague has the right to conclude a license agreement on the use of this
work as a school work pursuant to Section 60 paragraph 1 of the Copyright Act.

\vspace{10mm}

\hbox{\hbox to 0.5\hsize{%
In ........ date ............
\hss}\hbox to 0.5\hsize{%
signature of the author
\hss}}

\vspace{20mm}
\newpage

%%% Povinná informační strana diplomové práce

\vbox to 0.5\vsize{
\setlength\parindent{0mm}
\setlength\parskip{5mm}

Název práce:
Dotazování NoSQL databáz v prostředí MPS
% přesně dle zadání

Autor:
Bc. Radovan Duga

Katedra:  % Případně Ústav:
Katedra distribuovaných a spolehlivých systémů
% dle Organizační struktury MFF UK

Vedoucí diplomové práce:
RNDr. Pavel Parízek, Ph.D., Katedra distribuovaných a spolehlivých systémů
% dle Organizační struktury MFF UK, případně plný název pracoviště mimo MFF UK

Abstrakt:
% abstrakt v rozsahu 80-200 slov; nejedná se však o opis zadání diplomové práce
% TODO: cz change assignment -> abstract
S příchodem NoSQL databází se objevila i potřeba pro vznik doménově specifických dotazovacích
jazyků. Jednou ze zajímavých domén jsou grafové databáze jako například Neo4j s dotazovací jazykem
Cypher. Doménově specifické jazyky (DSLs) může být navržena a snadno použita pomocí speciálních
vývojových prostředích zvaných Language Workbenche. Velmi populární Language Workbench je MPS,
který implementuje koncept projekčních DSLs.

Tato práce zodpovídá otázku, zda Language Workbenche a projekční DSLs mohou být přínosem v
doméně NoSQL databází, vystihnout výhody projekčních DSLs použitím různých typů přístupu. Dalším
specifickým cílem je navrhnout a implementovat dotazovací DSL jazyk pro vybranou NoSQL databázi
(např. Neo4J nebo Redis) jako případová studie.

Klíčová slova:
NoSQL, MPS, dotaz, Cypher
% 3 až 5 klíčových slov

\vss}\nobreak\vbox to 0.49\vsize{
\setlength\parindent{0mm}
\setlength\parskip{5mm}

Title:
Querying NoSQL databases in MPS
% přesný překlad názvu práce v angličtině

Author:
Bc. Radovan Duga

Department:
Department of Distributed and Dependable Systems
% dle Organizační struktury MFF UK v angličtině

Supervisor:
RNDr. Pavel Parízek, Ph.D., Department of Distributed and Dependable Systems
% dle Organizační struktury MFF UK, případně plný název pracoviště
% mimo MFF UK v angličtině
sl
Abstract:
% abstrakt v rozsahu 80-200 slov v angličtině; nejedná se však o překlad
% zadání diplomové práce
% TODO: en change assignment -> abstract
With the advent of NoSQL databases, a need for targeted domain-specific query languages has become
evident. One of the interesting domains are graph databases, such as Neo4j with the query language
Cypher. Domain specific languages (DSLs) can be designed and easily used with the help of special
development environments called Language Workbenches. A very popular Language Workbench is MPS,
which implements the concept of projectional DSLs.

This work will answer the question whether Language Workbenches and projectional DSLs can make a
contribution in the domain of NoSQL databases, and identify the benefits of projectional DSLs over
different approaches. An additional specific goal is to design and implement a practical MPS-based
query DSL for a chosen NoSQL database (e.g., Neo4J or Redis) as a case study.

Keywords:
NoSQL, MPS, query, Cypher
% 3 až 5 klíčových slov v angličtině

\vss}